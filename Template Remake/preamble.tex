%%%%%%%%%%%%%%%%%%%%%%%
%       Imports       %
%%%%%%%%%%%%%%%%%%%%%%%

% Imports for document styling and formatting
\usepackage[tmargin=2cm,rmargin=1in,lmargin=1in,margin=0.85in,bmargin=2cm,footskip=.2in]{geometry} % for margins
\usepackage{xcolor} % for colors
\usepackage{bookmark} % for bookmarks
\usepackage{comment} % enables to use of multiline comments (\ifx \fi)
\usepackage{nameref} % for names
\usepackage{transparent} % for transparency
\usepackage[makeroom]{cancel} % for canceling terms
\usepackage{authblk} % for author affiliations
\usepackage{import} % for importing files
\usepackage{pdfpages} % for including pdfs
\setlength{\parindent}{1cm} % for paragraph indentation

\newcommand{\incfig}[1]{
    \def\svgwidth{\columnwidth}
    \import{./figures/}{#1.pdf_tex}
    } % for importing svg files

\hypersetup{
    pdftitle = {Title},
    colorlinks = true, linkcolor = doc!50!black, citecolor = doc!50!black, urlcolor = doc!50!black,
    bookmarksnumbered = true, bookmarksopen = true
} % hyperlink setup and metadata for the pdf
    
% Imports for Math formatting and symbols
\usepackage{amsmath, amssymb, amsthm, amsfonts, mathtools} % for math
\usepackage[varbb]{newpxmath} % for math fonts
\usepackage{xfrac} % for slanted fractions
\usepackage{tikzsymbols} % for symbols    
\renewcommand\qedsymbol{$\Laughey$} % for qed symbol

% Imports for images and diagrams 
\usepackage{graphicx} % for images
\usepackage{tikz} % for diagrams
\usepackage{tikz-cd} % for commutative diagrams

% Imports for lists, tables, columns, and boxes 
\usepackage{enumitem} % for lists
\usepackage{multicol, array} % for columns and arrays
\usepackage{varwidth} % for boxes
\usepackage[most, many, breakable]{tcolorbox} % for boxes

% Imports for code and algorithms
\usepackage{etoolbox} % for if statements
\usepackage{xifthen} % for if statements
\usepackage[ruled,vlined,linesnumbered]{algorithm2e} % for algorithms

\SetCommentSty{mycommfont} % for comments in algorithms
\newcommand\mycommfont[1]{\footnotesize\ttfamily\textcolor{blue}{#1}} % for comments in algorithms

% Imports for references and hyperlinks
\usepackage{hyperref,theoremref} % for hyperlinks and references

%%%%%%%%%%%%%%%%%%%%%%%
%        Colors       %
%%%%%%%%%%%%%%%%%%%%%%%

\definecolor{my_red}{HTML}{bd0000} % dark red
\definecolor{my_blue}{HTML}{001589} % dark blue
\definecolor{my_green}{HTML}{033b18} % dark green
\definecolor{my_purple}{HTML}{4c0088} % dark purple
\definecolor{my_gray}{HTML}{565656} % dark gray
\definecolor{my_yellow}{HTML}{b9a900} % dark yellow 
\definecolor{my_black}{HTML}{000000} % black

\definecolor{theorem_BG}{HTML}{F2F2F9} % light gray
\definecolor{theorem_F}{HTML}{00007B} % dark blue
\definecolor{corollary_BG}{HTML}{4c0088} % dark purple
\definecolor{corollary_F}{HTML}{000000} % black
\definecolor{lemma_BG}{HTML}{196800} % dark green
\definecolor{lemma_F}{HTML}{00091e} % deep dark blue
\definecolor{proposition_BG}{HTML}{005fe8} % dark blue
\definecolor{proposition_F}{HTML}{004246} % dark teal
\definecolor{exercise_BG}{HTML}{f6fcfc} % blank white 
\definecolor{exercise_F}{HTML}{417576} % deep teal
\definecolor{example_BG}{HTML}{f9f9f9} % light gray
\definecolor{example_F}{HTML}{000000} % black
\definecolor{example_TI}{HTML}{000000} % black

%%%%%%%%%%%%%%%%%%%%%%%
%     TCOLORBOXES     %
%%%%%%%%%%%%%%%%%%%%%%%

% Theorem Boxes % 
% Section Theorem 
\newtcbtheorem[number within=section]{section_theorem}{Theorem}
{
	enhanced,
	breakable,
	colback = theorem_BG,
	frame hidden,
	boxrule = 0sp,
	borderline west = {2pt}{0pt}{theorem_F},
	sharp corners,
	detach title,
	before upper = \tcbtitle\par\smallskip,
	coltitle = theorem_F,
	fonttitle = \bfseries\sffamily,
	description font = \mdseries,
	separator sign none,
	segmentation style={solid, theorem_F}
}
{theorem}

% Chapter Theorem
\newtcbtheorem[number within=chapter]{chapter_theorem}{Theorem}
{
	enhanced,
	breakable,
	colback = theorem_BG,
	frame hidden,
	boxrule = 0sp,
	borderline west = {2pt}{0pt}{theorem_F},
	sharp corners,
	detach title,
	before upper = \tcbtitle\par\smallskip,
	coltitle = theorem_F,
	fonttitle = \bfseries\sffamily,
	description font = \mdseries,
	separator sign none,
	segmentation style={solid, theorem_F}
}
{theorem}

% Corollery Boxes % 
% Section Corollary
\newtcbtheorem[number within=section]{section_corollary}{Corollary}
{
	enhanced,
	breakable,
	colback = corollary_BG!10,
	frame hidden,
	boxrule = 0sp,
	borderline west = {2pt}{0pt}{my_purple!85!corollary_F},
	sharp corners,
	detach title,
	before upper = \tcbtitle\par\smallskip,
	coltitle = corollary_BG!85!corollary_F,
	fonttitle = \bfseries\sffamily,
	description font = \mdseries,
	separator sign none,
	segmentation style={solid, corollary_BG!85!corollary_F}
}
{corollary}

% Chapter Corollary
\newtcbtheorem[number within=chapter]{chapter_corollary}{Corollary}
{
	enhanced,
	breakable,
	colback = corollary_BG!10,
	frame hidden,
	boxrule = 0sp,
	borderline west = {2pt}{0pt}{corollary_BG!85!corollary_F},
	sharp corners,
	detach title,
	before upper = \tcbtitle\par\smallskip,
	coltitle = corollary_BG!85!corollary_F,
	fonttitle = \bfseries\sffamily,
	description font = \mdseries,
	separator sign none,
	segmentation style={solid, corollary_BG!85!corollary_F}
}
{corollary}

% Lemma Boxes %
% Section Lemma
\newtcbtheorem[number within=section]{section_lemma}{Lemma}
{
	enhanced,
	breakable,
	colback = lemma_BG!10,
	frame hidden,
	boxrule = 0sp,
	borderline west = {2pt}{0pt}{lemma_F},
	sharp corners,
	detach title,
	before upper = \tcbtitle\par\smallskip,
	coltitle = lemma_F,
	fonttitle = \bfseries\sffamily,
	description font = \mdseries,
	separator sign none,
	segmentation style={solid, lemma_F}
}
{lemma}

% Chapter Lemma
\newtcbtheorem[number within=chapter]{chapter_lemma}{Lemma}
{
	enhanced,
	breakable,
	colback = lemma_BG!10,
	frame hidden,
	boxrule = 0sp,
	borderline west = {2pt}{0pt}{lemma_F},
	sharp corners,
	detach title,
	before upper = \tcbtitle\par\smallskip,
	coltitle = lemma_F,
	fonttitle = \bfseries\sffamily,
	description font = \mdseries,
	separator sign none,
	segmentation style={solid, lemma_F}
}
{lemma}

% Proposition Boxes %
% Section Proposition
\newtcbtheorem[number within=section]{section_proposition}{Proposition}
{
	enhanced,
	breakable,
	colback = proposition_BG!10,
	frame hidden,
	boxrule = 0sp,
	borderline west = {2pt}{0pt}{proposition_F},
	sharp corners,
	detach title,
	before upper = \tcbtitle\par\smallskip,
	coltitle = proposition_F,
	fonttitle = \bfseries\sffamily,
	description font = \mdseries,
	separator sign none,
	segmentation style={solid, proposition_F}
}
{proposition}

% Chapter Proposition
\newtcbtheorem[number within=chapter]{chapter_proposition}{Proposition}
{
	enhanced,
	breakable,
	colback = proposition_BG!10,
	frame hidden,
	boxrule = 0sp,
	borderline west = {2pt}{0pt}{proposition_F},
	sharp corners,
	detach title,
	before upper = \tcbtitle\par\smallskip,
	coltitle = proposition_F,
	fonttitle = \bfseries\sffamily,
	description font = \mdseries,
	separator sign none,
	segmentation style={solid, proposition_F}
}
{proposition}

% Claim Boxes %
% Section Claim 
\newtcbtheorem[number within=section]{section_claim}{Claim}
{%
	enhanced,
	breakable,
	colback = my_red!10,
	frame hidden,
	boxrule = 0sp,
	borderline west = {2pt}{0pt}{my_red},
	sharp corners,
	detach title,
	before upper = \tcbtitle\par\smallskip,
	coltitle = my_red!85!my_black,
	fonttitle = \bfseries\sffamily,
	description font = \mdseries,
	separator sign none,
	segmentation style={solid, my_red!85!my_black}
}
{claim}

% Chapter Claim
\newtcbtheorem[number within=chapter]{chapter_claim}{Claim}
{
	enhanced,
	breakable,
	colback = my_red!10,
	frame hidden,
	boxrule = 0sp,
	borderline west = {2pt}{0pt}{my_red},
	sharp corners,
	detach title,
	before upper = \tcbtitle\par\smallskip,
	coltitle = my_red!85!my_black,
	fonttitle = \bfseries\sffamily,
	description font = \mdseries,
	separator sign none,
	segmentation style={solid, my_red!85!my_black}
}
{claim}

% Exercise Boxes %
% Section Exercise  
\newtcbtheorem[number within=section]{section_exercise}{Exercise}
{%
	enhanced,
	breakable,
	colback = exercise_BG,
	frame hidden,
	boxrule = 0sp,
	borderline west = {2pt}{0pt}{exercise_F},
	sharp corners,
	detach title,
	before upper = \tcbtitle\par\smallskip,
	coltitle = exercise_F,
	fonttitle = \bfseries\sffamily,
	description font = \mdseries,
	separator sign none,
	segmentation style={solid, exercise_F},
}
{exercise}

% Chapter Exercise 
\newtcbtheorem[number within=chapter]{chapter_exercise}{Exercise}
{%
	enhanced,
	breakable,
	colback = exercise_BG,
	frame hidden,
	boxrule = 0sp,
	borderline west = {2pt}{0pt}{exercise_F},
	sharp corners,
	detach title,
	before upper = \tcbtitle\par\smallskip,
	coltitle = exercise_F,
	fonttitle = \bfseries\sffamily,
	description font = \mdseries,
	separator sign none,
	segmentation style={solid, exercise_F},
}
{exercise}

% Example Boxes %
% Section Example 
\newtcbtheorem[number within=section]{section_example}{Example}
{%
	colback = example_BG,
	breakable,
	colframe = example_F,
	coltitle = example_TI,
	boxrule = 1pt,
	sharp corners,
	detach title,
	before upper=\tcbtitle\par\smallskip,
	fonttitle = \bfseries,
	description font = \mdseries,
	separator sign none,
	description delimiters parenthesis
}
{example}

% Chapter Example
\newtcbtheorem[number within=chapter]{chapter_example}{Example}
{%
	colback = example_BG,
	breakable,
	colframe = example_F,
	coltitle = example_TI,
	boxrule = 1pt,
	sharp corners,
	detach title,
	before upper=\tcbtitle\par\smallskip,
	fonttitle = \bfseries,
	description font = \mdseries,
	separator sign none,
	description delimiters parenthesis
}
{example}

% Definition Boxes %
% Section Definition
\newtcbtheorem[number within=section]{section_definition}{Definition}
{
    enhanced,
	before skip = 2mm,
    after skip = 2mm, 
    colback = red!5,
    colframe = red!80!black,
    boxrule = 0.5mm,
	attach boxed title to top left = 
    {
        xshift = 1cm,
        yshift* = 1mm-\tcboxedtitleheight,
    }, 
    varwidth boxed title* = -3cm,
	boxed title style = 
    {
        frame code = 
        {
					\path[fill = tcbcolback]
                    ([yshift = -1mm, xshift = -1mm]frame.north west)
					arc[start angle = 0, end angle = 180, radius = 1mm]
					([yshift = -1mm, xshift = 1mm]frame.north east)
					arc[start angle = 180, end angle = 0, radius = 1mm];
					\path[left color = tcbcolback!60!black, right color = tcbcolback!60!black,
						middle color = tcbcolback!80!black]
					([xshift = -2mm]frame.north west)-- 
                    ([xshift = 2mm]frame.north east)[rounded corners = 1mm]-- 
                    ([xshift = 1mm, yshift = -1mm]frame.north east)--
					(frame.south east)-- 
                    (frame.south west)--
					([xshift = -1mm, yshift = -1mm]frame.north west)[sharp corners]-- 
                    cycle;
        },
        interior engine = empty,
    },
	fonttitle = \bfseries,
	title = {#2},
    #1
}{definition}

% Chapter Definition
\newtcbtheorem[number within=chapter]{chapter_definition}{Definition}
{
    enhanced,
	before skip = 2mm,
    after skip = 2mm, 
    colback = red!5,
    colframe = red!80!black,
    boxrule = 0.5mm,
	attach boxed title to top left = 
    {
        xshift = 1cm,
        yshift* = 1mm-\tcboxedtitleheight,
    }, 
    varwidth boxed title* = -3cm,
	boxed title style = 
    {
        frame code = 
        {
					\path[fill = tcbcolback]
                    ([yshift = -1mm, xshift = -1mm]frame.north west)
					arc[start angle = 0, end angle = 180, radius = 1mm]
					([yshift = -1mm, xshift = 1mm]frame.north east)
					arc[start angle = 180, end angle = 0, radius = 1mm];
					\path[left color = tcbcolback!60!black, right color = tcbcolback!60!black,
						middle color = tcbcolback!80!black]
					([xshift = -2mm]frame.north west)-- 
                    ([xshift = 2mm]frame.north east)[rounded corners = 1mm]-- 
                    ([xshift = 1mm, yshift = -1mm]frame.north east)--
					(frame.south east)-- 
                    (frame.south west)--
					([xshift = -1mm, yshift = -1mm]frame.north west)[sharp corners]-- 
                    cycle;
        },
        interior engine = empty,
    },
	fonttitle = \bfseries,
	title = {#2},
    #1
}{definition}


\newtcbtheorem{question}{Question}{enhanced,
	breakable,
	colback=white,
	colframe=myb!80!black,
	attach boxed title to top left={yshift*=-\tcboxedtitleheight},
	fonttitle=\bfseries,
	title={#2},
	boxed title size=title,
	boxed title style={%
			sharp corners,
			rounded corners=northwest,
			colback=tcbcolframe,
			boxrule=0pt,
		},
	underlay boxed title={
			\path[fill=tcbcolframe] (title.south west)--
            (title.south east) to[out=0, in=180] ([xshift=5mm]title.east)--
			(title.center-|frame.east)
			[rounded corners=\kvtcb@arc] |-
			(frame.north) -| cycle;
		},
	#1
}{question}